
We present here an outline of the derivation of the homology relations. We assume the following relations to be our starting point:
\begin{align}
    L      & \propto M^3 \kappa \ ,                                   \\
    T_c    & \propto M^{\frac{4}{n+3}} \kappa^{\frac{-1}{n+3}} \ ,    \\
    \rho_c & \propto M^{\frac{6-2n}{n+3}} \kappa^{\frac{-3}{n+3}} \ .
\end{align}
The opacity can be either
\begin{align}
    \kappa =
    \begin{cases}
        \kappa_{\rm Kr} \propto \rho T^{\frac{-7}{2}} \  & \text{(Kramers)}                     \\
        \kappa_{\rm es} \equiv {\rm cte}                 & \text{(e$^{-} \ \text{scattering}$)} \\
    \end{cases}
\end{align}
If the opacity follows an electron scattering law, it is constant and it is trivial to see that from Eqs (17, 18, 19) we arrive at Eqs (10, 11, 12) in Sec. 3. If the opacity follows a Kramers law, we have to find an expression
\begin{align}
    \kappa_{\rm Kr} \propto M^{\frownie}\ ,
\end{align}
where $\frownie$ is a function of $n$. That must be done by replacing $\rho$ and $T$ with adequate homology relations in a way that we are left only with dependence on $M$. We spare ourselves the pain of doing this here. After some algebraic juggling we arrive at
\begin{equation}
    \kappa_{\rm Kr} \propto M^{-\frac{4n+16}{5+2n}} \ .
    \label{eq:eq}
\end{equation}
Injecting this expression into Eqs (17, 18, 19) we arrive at Eqs (7, 8, 9) in Sec. 3.